\documentclass{llncs}
%
\begin{document}
\title{Optimizing Abstract$^2$ Machines}
%
\frontmatter          % for the preliminaries
%
\author{J.~Ian Johnson\inst{1} \and Matthew Might\inst{2} \and David Van Horn\inst{1}}
\institute{Northeastern University \and University of Utah}

\maketitle              % typeset the title of the contribution

\begin{abstract}
The abstracting abstract machines approach to program analysis makes
it easy to design and implement sound and computable abstract
interpretations of higher-order programs.
%
However, the straightforward implementation is generally
not very fast.
%
This article contributes a step by step process for going from a naive
abstract$^2$ machine to an efficient program analyzer that is two
orders of magnitude faster than the starting point, yet just as
precise.
\end{abstract}
%
\section{Introduction}

\section{Starting point}
\subsection{An abstract$^2$ machine for PCF}
\subsection{An example: Analyzing Church number computations}
\subsection{Store widening}
\subsection{Baseline evaluation}

\section{Precision preserving recipe}
\subsection{Fixed-point specialization}
\subsection{Lazy non-determinism}
\subsection{Abstract compilation}
\subsection{Computing with store changes}

\section{Evaluation}

\section{Related work}

\section{Conclusion}

Abstract machines are not only a good model for rapid analysis
development, they also can be systematically developed into efficient
algorithms.

\end{document}
