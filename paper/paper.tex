\documentclass{llncs}
\usepackage{alltt}
\usepackage{amsmath}
\usepackage{amssymb}
\usepackage{xltxtra}
\setmonofont[Scale=MatchLowercase]{DejaVu Sans Mono}
%
\newcommand{\superscript}[1]{\ensuremath{^{#1}}}
\newcommand{\tuple}[2]{{\tt #1\ }({#2})}

% Values
\newcommand{\clos}[1]{\tuple{clos}{#1}}
\newcommand{\rlos}[1]{\tuple{rlos}{#1}}

% States
\newcommand{\ev}[1]{\tuple{ev}{#1}}
\newcommand{\co}[1]{\tuple{co}{#1}}
\newcommand{\ap}[1]{\tuple{ap}{#1}}
\newcommand{\ans}[1]{\tuple{ans}{#1}}

% Continuations
\newcommand{\kmt}{\tt mt}
\newcommand{\kar}[1]{\tuple{ar}{#1}}
\newcommand{\kfn}[1]{\tuple{fn}{#1}}
\newcommand{\kif}[1]{\tuple{fi}{#1}}
\newcommand{\kuop}[1]{\tuple{oa}{#1}}
\newcommand{\kbopa}[1]{\tuple{oa1}{#1}}
\newcommand{\kbopb}[1]{\tuple{oa2}{#1}}

% Syntax
\newcommand{\syntax}[1]{{\tt #1}}
\newcommand{\sapp}[2]{\tuple{app}{#1,#2}}
\newcommand{\slam}[2]{\tuple{lam}{#1,#2}}
\newcommand{\srec}[3]{\tuple{rec}{#1,#2,#3}}
\newcommand{\svar}[1]{\tuple{var}{#1}}
\newcommand{\snum}[1]{\tuple{num}{#1}}
\newcommand{\sif}[3]{\tuple{if}{#1,#2,#3}}
\newcommand{\sopu}[2]{\tuple{op}{#1,#2}}
\newcommand{\sopb}[3]{\tuple{op2}{#1,#2,#3}}
\newcommand{\strue}{{\tt tt}}
\newcommand{\sfalse}{{\tt ff}}
\newcommand{\saddone}{{\tt add1}}
\newcommand{\ssubone}{{\tt sub1}}
\newcommand{\szerohuh}{\syntax{zero?}}
\newcommand{\szero}{\syntax{0}}

\newcommand{\sNum}{\syntax{Z}}

% Metavariables
\newcommand{\maddr}{a}
\newcommand{\mvar}{x}
\newcommand{\mvarf}{f}
\newcommand{\mexp}{e}
\newcommand{\mexpi}[1]{e_{#1}}
\newcommand{\mexpf}{f}
\newcommand{\menv}{\rho}
\newcommand{\mkont}{\kappa}
\newcommand{\msto}{\sigma}
\newcommand{\mop}{o}
\newcommand{\mval}{v}
\newcommand{\mnum}{z}
\newcommand{\mvalx}[1]{#1}

\newcommand\machstep{\longmapsto}

\begin{document}
\title{Optimizing Abstract$^2$ Machines}
%
\frontmatter          % for the preliminaries
%
\author{J.~Ian Johnson\inst{1} \and Matthew Might\inst{2} \and David Van Horn\inst{1}}
\institute{Northeastern University \and University of Utah}

\maketitle              % typeset the title of the contribution

\begin{abstract}
The abstracting abstract machines approach to program analysis makes
it easy to design and implement sound and computable abstract
interpretations of higher-order programs.
%
However, the straightforward implementation is generally
not very fast.
%
This article contributes a step by step process for going from a naive
abstract$^2$ machine to an efficient program analyzer that is (nearly)
three orders of magnitude faster than the starting point, yet just as
precise (and purely functional).
\end{abstract}
%
\section{Introduction}

\subsection{Notation and prerequisites}

Prereqs: Semantics Engineering, AAM.

Notation: concrete examples of programs under analysis is given in
(monochrome) Scheme notation.  Code is given in (syntax colored)
Racket.

\newpage
\section{Starting point}
\subsection{An abstract$^2$ machine for ISWIM}

\begin{figure}
\begin{align*}
\ev{\svar{x},\rho,\kappa,\sigma} &\machstep
\co{\kappa,v,\sigma}
\text{ where }v \in \msto(\menv(x))
\\
\ev{\snum{n},\menv,\mkont,\msto} &\machstep
\co{\mkont,n,\msto}
\\
\ev{\slam\mvar\mexp,\menv,\mkont,\msto} &\machstep
\co{\mkont,\clos{\mvar,\mexp,\menv}}
\\
%% \ev{\srec\mvarf\mvar\mexp,\menv,\mkont,\msto} &\machstep
%% \co{\mkont,\rlos{\mvarf,\mvar,\mexp,\menv}}
%% \\
\ev{\sapp{\mexpi0}{\mexpi1},\menv,\mkont,\msto} &\machstep
\ev{\mexpi{0},\menv,\kar{\mexpi{1},\menv,a},\msto'}
\text{ where }a,\msto' = \mathit{push}(l,\msto,\mkont)
\\
\ev{\sif{\mexpi0}{\mexpi1}{\mexpi2},\menv,\mkont,\msto} &\machstep
\ev{\mexpi0,\menv,\kif{\mexpi1,\mexpi2,\menv,\maddr},\msto'}
\text{ where }a,\msto' = \mathit{push}(l,\msto,\mkont)
\\
\ev{\sopu\mop\mexp,\menv,\mkont,\msto} &\machstep
\ev{\mexp,\menv,\kuop{\mop,\maddr},\msto'}
\text{ where }a,\msto' = \mathit{push}(l,\msto,\mkont)
%%\\
%% \ev{\sopb\mop{\mexpi0}{\mexpi1},\menv,\mkont,\msto} &\machstep
%% \ev{\mexpi0,\menv,\kbopa{\mexpi1,\menv,\maddr},\msto'}
%% \text{ where }a,\msto' = \mathit{push}(l,\msto,\mkont)
\\[2mm]
\co{\kmt,\mval,\msto} &\machstep
\ans{\msto,\mval}
\\
\co{\kar{\mexp,\menv,\maddr},\mval,\msto} & \machstep
\ev{\mexp,\menv,\kfn{\mval,\maddr},\msto}
\\
\co{\kfn{{\mvalx{u}},\maddr},\mval,\msto} & \machstep
\ap{\mval,\mvalx{u},\mkont,\msto}
\text{ where }\mkont \in \msto(\maddr)
\\
\co{\kif{\mexpi0,\mexpi1,\menv,\maddr},\strue,\msto} & \machstep
\ev{\mexpi0,\menv,\mkont,\msto}
\text{ where }\mkont\in\msto(\maddr)
\\
\co{\kif{\mexpi0,\mexpi1,\menv,\maddr},\sfalse,\msto} & \machstep
\ev{\mexpi1,\menv,\mkont,\msto}
\text{ where }\mkont\in\msto(\maddr)
\\
\co{\kuop{\mop,\maddr},\mval,\msto} & \machstep
\co{\mkont,\mval',\msto}
\text{ where }\mkont\in\msto(\maddr)
\text{ and } \mval'\in\delta(\mop,\mval)
%% \\
%% \co{\kbopa{\mop,\mexp,\menv,\maddr},\mval,\msto} & \machstep
%% \ev{\mexp,\menv,\kbopb{\mop,\mval,\maddr},\msto}
%% \\
%% \co{\kbopb{\mop,\mvalx{u},\maddr},\mval,\msto} & \machstep
%% \co{\mkont,\mval',\msto}
%% \text{ where }\mkont=\msto(\maddr)
%% \text{ and } \mval'=\delta(\mop,\mval,\mvalx{u})
\\[2mm]
\ap{\clos{\mvar,\mexp,\menv},\mval,\mkont} & \machstep
\ev{\mexp,\menv',\mkont,\msto'}
\text{ where }\menv',\msto' = \mathit{bind}(\msto,\mvar,\mval)
\\
%% \ap{\rlos{\mvarf,\mvar,\mexp,\menv},\mval,\mkont} & \machstep
%% \ev{\mexp,\menv',\mkont,\msto'}
%% \text{ where }\menv',\msto' = \mathit{bind}(\msto,\langle\mvar,\mval\rangle,\langle\mvarf,
%% \rlos{\mvarf,\mvar,\mexp,\menv}\rangle)
\end{align*}
\caption{Abstract$^2$ machine for ISWIM}
\label{fig:aam}
\end{figure}

\paragraph{Concrete interpretation:}

To obtain a concrete interpreter from the relation given in
Figure~\ref{fig:aam}, set the implicit parameters as follows:
\begin{gather}
\begin{align*}
\mathit{push}(l,\msto,\mkont) &= \maddr,\sigma[\maddr\mapsto\{\mkont\}]
\mbox{ where }\maddr \notin\msto
\\
\mathit{bind}(\msto,\mvar,\mval) &= \menv[\mvar\mapsto\maddr],\sigma[\maddr\mapsto\{\mval\}]
\end{align*}
\\[2mm]
\begin{align*}
\mnum+1 &\in \delta(\saddone,\mnum) &
\mnum-1 &\in \delta(\ssubone,\mnum)\\
\strue &\in \delta(\szerohuh,\szero) &
\sfalse &\in \delta(\szerohuh,\mnum)\text{ if }\mnum\neq \szero\\
\end{align*}
\end{gather}

\paragraph{Abstract interpretation:}

To obtain an abstract interpreter from the relation given in
Figure~\ref{fig:aam}, set the implicit parameters as follows:
\begin{gather}
\begin{align*}
\widehat{\mathit{push}}(l,\msto,\mkont) &= l,\sigma\sqcup[l\mapsto\{\mkont\}]
\mbox{ where }\maddr \notin\msto
\\
\widehat{\mathit{bind}}(\msto,\mvar,\mval) &= \menv[\mvar\mapsto\mvar],\msto\sqcup[\mvar\mapsto\{\mval\}]
\end{align*}
\\[2mm]
\begin{align*}
\sNum &\in \hat\delta(\saddone,\mnum) &
\sNum &\in \hat\delta(\ssubone,\mnum)\\
\strue &\in \hat\delta(\szerohuh,\sNum) &
\sfalse &\in \hat\delta(\szerohuh,\sNum)\\
\strue &\in \hat\delta(\szerohuh,\szero) &
\sfalse &\in \hat\delta(\szerohuh,\mnum)\text{ if }\mnum\neq \szero\\
\end{align*}
\end{gather}


\begin{figure}
\begin{alltt}
  ;; State \(\rightarrow\) Setof State
  (define (step state)
    (match state
      [(ev σ e ρ k)
       (match e
         [(var l x)           (for/set ((v (lookup ρ σ x))) (co σ k v))]
         [(num l n)           (set (co σ k n))]
         [(bln l b)           (set (co σ k b))]
         [(lam l x e)         (set (co σ k (clos l x e ρ)))]
         [(rec f (lam l x e)) (set (co σ k (rlos l f x e ρ)))]
         [(app l f e)
          (define-values (σ* a) (push state))
          (set (ev σ* f ρ (ar e ρ a)))]
         [(ife l e0 e1 e2)
          (define-values (σ* a) (push state))
          (set (ev σ* e0 ρ (ifk e1 e2 ρ a)))]
         [(1op l o e)
          (define-values (σ* a) (push state))
          (set (ev σ* e ρ (1opk o a)))])]
      [(co σ k v)
       (match k
         ['mt (set (ans σ v))]
         [(ar e ρ l) (set (ev σ e ρ (fn v l)))]
         [(fn f l)   (for/set ((k (get-cont σ l))) (ap σ f v k))]
         [(fi c a ρ l)
          (for/set ((k (get-cont σ l)))
            (ev σ (if v c a) ρ k))]
         [(oa o l)
          (for*/set ((k (get-cont σ l))
                     (v (delta o (list v))))
            (co σ k v))]
      [(ap σ fun a k)
       (match fun
         [(clos l x e ρ)
          (define-values (ρ* σ*) (bind state))
          (set (ev σ* e ρ* k))]
         [_ (set)])]
\end{alltt}
\caption{Implementation of machine transition relation.}
\end{figure}

\subsection{An example: Analyzing Church number computations}

\newcommand{\church}[1]{\(\ulcorner{\tt #1}\urcorner\)}

\begin{alltt}
;; multiplication distributes over addition
((church=? ((mult \church2) ((plus \church1) \church3)))
 ((plus ((mult \church2) \church1)) ((mult \church2) \church3)))))
\end{alltt}

Where \syntax{church=?} is an equality function for Church numerals
written in terms of recursion and an iterative \syntax{pred} function.
The \syntax{plus} and \syntax{mult} functions are written as usual.

This program reaches $N$ states under concrete interpretation.

\subsection{Store widening}

\subsection{Baseline evaluation}

Wide Store: cpu time: 551571 real time: 571319 gc time: 4003

\section{Precision preserving recipe}

\subsection{Lazy non-determinism}

Lazy:
   cpu time: 32481 real time: 32881 gc time: 547

\subsection{Abstract compilation}

Compile:
   cpu time: 255397 real time: 261532 gc time: 2947

\noindent
Compile + Lazy:
   cpu time: 31173 real time: 31642 gc time: 739

\subsection{Fixed-point specialization}

%% Should be commutted to beginning

Special + Compile + Lazy:
   cpu time: 14212 real time: 14681 gc time: 823

\subsection{Computing with store changes}

Delta Store + Special + Compile + Lazy:
   cpu time: 668 real time: 686 gc time: 41

\section{Evaluation}

Other benchmarks

\section{Related work}

\section{Conclusion}

Abstract machines are not only a good model for rapid analysis
development, they also can be systematically developed into efficient
algorithms.

\end{document}
